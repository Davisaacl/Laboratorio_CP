\documentclass[a4paper]{article}

%% Language and font encodings
%\usepackage{fontspec}

%% Sets page size and margins
\usepackage[a4paper,top=3cm,bottom=2cm,left=3cm,right=3cm,marginparwidth=1.75cm]{geometry}

%% Useful packages
\usepackage{amsmath,amsthm,amssymb,amsfonts}
\usepackage{graphicx}
\usepackage[colorinlistoftodos]{todonotes}
\usepackage[colorlinks=true, allcolors=blue]{hyperref}
\usepackage{bbm}

\title{Laboratorio 1. Espacios de probabilidad}

\author{Cálculo de Probabilidades I\\
        Departamento de Estadística\\
}

\date{20/01/2021}

\begin{document}
\maketitle

\section*{Ejercicios}

\begin{enumerate}
\item Una varilla de metal de longitud $l$ se rompe en dos puntos elegidos al azar ¿Cuál es la probabilidad de que los tres elementos así obtenidos formen un triángulo?

\item Se escoge un número al azar dentro del intervalo $(-1,1)$ ¿Cuál es la probabilidad de que la ecuación cuadrática $ax^2 + x + 1 = 0$ tenga dos raíces reales?

\item Un modelo de una ruleta puede construirse tomando un espacio de probabilidad uniforme sobre una circunferencia de radio $1$, de manera que la probabilidad de que el apuntador caiga en una arco de longitud $s$ es $s/2\pi$. Suponga que el círculo se divide en 37 zonas numeradas $1, 2, . . . , 37$. Calcule la probabilidad de que la ruleta caiga en una zona par.

\item Suponga que se elige un punto al azar del cuadrado unitario. Sea A el evento que determinado por el triángulo formado por las líneas $y = 0, x = 1, x = y$, y sea $B$ el evento definido por el rectángulo de vértices $(0, 0), (1, 0), (1, 1/2), (0, 1/2)$. Calcule $P(A \cap B)$ y $P(A \cup B)$.

\item Una caja tiene 10 bolas numeradas $1, 2, . . . , 10$. Una bola se elige al azar y una segunda bola se elige de las 9 restantes. Encuentre la probabilidad de que los númeross de las 2 bolas difiera en 2 o más.

\item En un bosque hay 20 alces, de los cuales 5 fueron capturados, marcados y liberados. Cierto tiempo después, 4 de los 20 alces fueron capturados ¿Cuál es la probabilidad de que 2 de estos 4 alces hayan sido marcados anteriormente? ¿Qué supuesto está haciendo para sus cálculos?

\item Suponga que un experimento se realiza n veces. Para cualquier evento $E$ del espacio muestral, sea $n(E)$ el número de veces que $E$ ocurre, y defina $f(E) = n(E)/n$. Muestre que f es una medida de probabilidad. (Es decir, satisface los Axiomas de la Probabilidad.)

\item Si $P(E) = 0.9$ y $P(F) = 0.8$, muestre que $P(E \cap F ) \ge 0.7.$ En general, pruebe la \textit{desigualdad de Bonferroni}, 

$$ \mathbbm{P}(E \cap F) \ge \mathbbm{P}(E) + \mathbbm{P}(F) - 1$$

\item  Sea $(\Omega, S,\mathbbm{P})$ un espacio de probabilidad donde $S$ es una $\sigma$-algebra de subconjuntos de $\Omega$ y $\mathbbm{P}$ es una medida de probabilidad que asigna probabilidad $p(> 0)$ a cada uno de los puntos de $\Omega$.

\begin{itemize}
    \item Muestre que $\Omega$ debe tener un número finito de puntos. (\textbf{Sugerencia: muestre que $\Omega$ no puede tener más de $1/p$ puntos}.)
    \item Muestre que si $n$ es el número de puntos de $\Omega$, entonces p debe ser $1/n$.
\end{itemize}
\end{enumerate}
\newpage

\section*{Soluciones}
\begin{enumerate}
    \item Este problema se resuelve a través del enfoque geométrico de la probabilidad. Empezamos definiendo el conjunto $\Omega = \{ (x,y): 0 < x, y < l \}.$
    
    Supongamos que $0 < x < y < l$, entonces la varilla de metal se puede dividir en tres segmentos: $x, y-x, l-y$. Con conocimientos previos de trigonometría, sabemos que se puede formar un triángulo syss 
    
    \[ x < (y-x)+(l-y) \Rightarrow x < l/2, \]
    \[y-x < x+(l-y) \Rightarrow y < l/2 + x, \]
    \[l-y < x + (y-x) \Rightarrow y > l/2 \]
    
    Graficando estas últimas desigualdades en un plano x-y, nos damos cuenta que, si $A$ es el evento de interés, entonces:
    
    \[ \mathbb{P}(A) = \frac{Area(A)}{Area(\Omega)} = \frac{1}{4}\]
    
    \item Sabemos que $ax^2 + x + 1 = 0$ tiene raíces reales si $1-4(a)(1) \ge 0 \Rightarrow a \le 1/4.$ Ahora bien, como $\Omega = (-1,1)$ y sea $A = \{ a \in \Omega: 1-4a \le 0 \} = (-1, 1/4]$ nuestro evento de interés. Por probabilidad clásica:
    
    \[ \mathbb{P}(A) = \frac{longitud(A)}{longitud(\Omega)} = \frac{5/4}{2} = \frac{5}{8}\]
    
    \item En primer lugar, definimos al evento $E$ como el evento cuando el apuntador cae en zona par, es decir $E = \{2,4,6,...,36\}.$ Por el enfoque de probabilidad clásica, podemos dividir casos favorables entre casos totales, luego: 
    
    \[ \mathbb{P}(E) = \frac{pares}{total} = \frac{18}{37}\]
    
    \item Graficando las línes en el cuadrado unitario, la solución resulta ser el área de interés vs el área total:
    
    \[\mathbb{P}(A \cap B) = \frac{1/4 + 1/8}{1} = 3/8\]
    \[\mathbb{P}(A \cup B) = \mathbb{P}(A) + \mathbb{P}(B) - \mathbb{P}(A \cap B) = 1/2 + 1/2 - 3/8 = 5/8 \]\
    
    \item Sea E la probabilidad de que los números de las bolas difieran en 2 o más. Notamos que el número de casos totales es $| \Omega | = \binom{10}{1} \binom{9}{1}$. Para el caso del número de casos favorables, definamos una tupla de la forma $(x,y)$ donde $x$ es el número de la primera extracción, mientras que $y$ es el número de la segunda extracción. De esta forma, lo único que se tiene que hacer es fijar alguno de los dos números, y llenar aquellos que difieran en dos o más. Para el caso, por ejemplo, de $x = 1$, tenemos el siguiente conjunto:
    
    $$ \{(1,3),(1,4),...,(1,10)\}$$
    
    Repitiendo el ejercicio anterior para $i=1,...,10$ obtenemos que el número de casos favorables es $ | E | = 72.$ Por lo tanto, por probabilidad clásica, llegamos a que $P(E) = \frac{72}{10 \cdot 9}$

    \item Es fácil ver que podemos dividir a nuestra población total de alces en dos subgrupos más pequeños: los alces marcados, y los no marcados. Ahora bien, definamos el evento $A$ como el número de alces marcados. Luego, nos interesa obtener la probabilidad de que suceda el evento $\{A = 2\}$. Por probabilidad clásica, obtenemos:
    
    $$ P(\{A = 2\}) = \frac{\binom{5}{2} \binom{15}{2}}{\binom{20}{4}}$$
    
    Vale la pena mencionar que este problema se puede resolver con ayuda de la distribución hipergeométrica. 
    
    \item Para este problema, debemos desarrollar los tres axiomas de Kolmogorov:
    
    \begin{enumerate}
        \item $f(E)= \frac{n(E)}{n} \ge 0$. Ya que, si el experimento no ocurre, entonces $n(E)=0$; por otro lado, si el experimento ocurre, entonces $n(E) \ge 0.$
        \item \[ f(\Omega) = f(E \cup E^c) = f(E) + f(E^c) = \frac{n(E)}{n} + \frac{1-n(E)}{n} = 1 \]
        \item \[ f(\bigcup_{i=1}^\infty E_i) = \frac{n(\bigcup_{i=1}^\infty E_i)}{n} = \sum_{i=1}^\infty \frac{n(E_i)}{n} = \sum_{i=1}^\infty f(E_i) \]
    \end{enumerate}
    
    \item Vamos a probar directamente la \textit{desigualdad de Bonferrioni}. Como $\mathbb{P}(E \cup F)$ es una medida de probabilidad, se cumple que $\mathbb{P}(E \cup F) \in (0,1),$
    
    \[ \Rightarrow \mathbb{P}(E) + \mathbb{P}(F) - \mathbb{P}(E \cap F) \le 1\]
    \[ \Rightarrow \mathbb{P}(E \cap F) \ge \mathbb{P}(E) + \mathbb{P}(F) -1 \]
    
    \item Sea $\Omega = \{ \omega_1, \omega_2, ..., \omega_n\}$ con $\mathbb{P}(\{\omega_i\}) = p (>0)$ $i = 1,...,n.$ Por segundo axioma de Kolmogorov, sabemos que:
    
    \[ \mathbb{P}(\Omega) = \mathbb{P}(\{\omega_1,...,\omega_n \}) = \mathbb{P}(\cup_{i = 1}^n \{\omega_i\}) = \sum_{i=1}^n \mathbb{P}(\{ \omega_i\}) = n \cdot p = 1\,... \,(*)\]
    
    Si suponemos ahora que $|\Omega| > 1/p (= n)$
    
    \[\Rightarrow \lim_{n \to +\infty} \mathbb{P}(\Omega) = \lim_{n \to +\infty} n \cdot p \neq 1 !\]
    
    Lo cual es un absurdo. Por otro lado, para el segundo inciso, tomando en cuenta la igualdad (*), concluimos que $p = 1/n.$
    
    
    
\end{enumerate}

\end{document}