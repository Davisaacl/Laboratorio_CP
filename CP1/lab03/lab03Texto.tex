\section{Técnicas de conteo}

\begin{enumerate}

	
\item ¿Cuántas placas diferentes en la Ciudad de México se pueden formar si los primeros tres lugares serán ocupados por letras (26) y los siguientes cuatro se ocuparán por números del 0 al 9?

\item A partir del inciso anterior. ¿Cuántas placas se pueden formar si no se permite la repetición de números y letras?

\item ¿De cuántas maneras pueden sentarse 10 personas en un banco si hay 4 sitios disponibles?

\item Se debe colocar a 5 hombres y 5 mujeres en una fila de modo que las mujeres ocupen los lugares pares. ¿De cuántas maneras puede hacerse?

\item ¿De cuántas maneras se pueden sentar 5 mujeres y 5 hombres alrededor de una mesa redonda, si deben sentarse alternadamente?

\item Sean $A$ y $B$ dos conjuntos finitos tales que $|A|=m$ y $|B|=n$.
\begin{enumerate}
	\item ¿Cuántas funciones diferentes $f:A \rightarrow B$ se pueden definir?
	\item ¿Cuántas funciones inyectivas distintas $f:A \rightarrow B$ se pueden construir?
\end{enumerate}

\item Un alumno de Cálculo de Probabilidades I debe escoger 7 de las 10 preguntas del examen final departamental.
\begin{enumerate}
	\item ¿De cuántas maneras puede elegir?
	\item Si las primeras 4 son obligatorias, ¿cuántas formas le quedan para escoger?
\end{enumerate}

\item Cuatro libros de matemáticas, seis de física y dos de quíica han de ser colocados en una estantería. ¿Cuántas colocaciones diferentes se admiten para cada caso?
\begin{enumerate}
	\item Los libros de cada materia han de estar juntos.
	\item Solo los libros de matemáticas tienen que estar juntos.
\end{enumerate}

\item Determine el número de distintos arreglos $(x_1, . . . , x_n)$, tales que $x_i$ es 0 ó 1 y

\[
\sum_{i=1}^{n} x_i \ge k 
\]

\item ¿Cuántas subconjuntos existen de un conjunto de $n$ elementos? (\textbf{Sugerencia:} Defina a las combinaciones de $n$ en $k$ como el número de subconjuntos de tamaño $k$ y aplique el teorema del binomio de Newton.)

\item ¿Cuántas formas hay de tener cierta mano en el juego de póquer, para los siguientes casos? (Suponga que no se utilizan las cartas denominadas \textit{Joker})
\begin{enumerate}
	\item No se tiene dos cartas del mismo número.
	\item \textit{Full house} (Consta de una tercia -tres números iguales- y un par -dos números iguales-).
	\item Póquer (cuatro números iguales).
	\item Color (Cinco cartas del mismo palo).
\end{enumerate}

\item ¿De cuántas formas se pueden ordenar las palabras siguientes?
\begin{enumerate}
	\item ABRACADABRA
	\item SUPERCALIFRAGILÍSTICO
\end{enumerate}

\end{enumerate}

\subsubsection*{Textos de apoyo.} 
\citeAY{Bartle:1966}; \citeAY{Blitzstein&Hwang:2014}; \citeAY{Hoel&etal:1971}; \citeAY{Rincon:2014}; \citeAY{Ross:2014}.
