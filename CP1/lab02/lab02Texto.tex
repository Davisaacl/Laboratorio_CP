\section{Espacios de Probabilidad}

\begin{enumerate}
		\item Una varilla de metal de longitud $l$ se rompe en dos puntos elegidos al azar ¿Cuál es la probabilidad de que los tres elementos así obtenidos formen un triángulo?

\item Se escoge un número al azar dentro del intervalo $(-1,1)$ ¿Cuál es la probabilidad de que la ecuación cuadrática $ax^2 + x + 1 = 0$ tenga dos raíces reales?

\item Un modelo de una ruleta puede construirse tomando un espacio de probabilidad uniforme sobre una circunferencia de radio $1$, de manera que la probabilidad de que el apuntador caiga en una arco de longitud $s$ es $s/2\pi$. Suponga que el círculo se divide en 37 zonas numeradas $1, 2, . . . , 37$. Calcule la probabilidad de que la ruleta caiga en una zona par.

\item Suponga que se elige un punto al azar del cuadrado unitario. Sea A el evento que determinado por el triángulo formado por las líneas $y = 0, x = 1, x = y$, y sea $B$ el evento definido por el rectángulo de vértices $(0, 0), (1, 0), (1, 1/2), (0, 1/2)$. Calcule $P(A \cap B)$ y $P(A \cup B)$.

\item Una caja tiene 10 bolas numeradas $1, 2, . . . , 10$. Una bola se elige al azar y una segunda bola se elige de las 9 restantes. Encuentre la probabilidad de que los númeross de las 2 bolas difiera en 2 o más.

\item En un bosque hay 20 alces, de los cuales 5 fueron capturados, marcados y liberados. Cierto tiempo después, 4 de los 20 alces fueron capturados ¿Cuál es la probabilidad de que 2 de estos 4 alces hayan sido marcados anteriormente? ¿Qué supuesto está haciendo para sus cálculos?

\item Suponga que un experimento se realiza n veces. Para cualquier evento $E$ del espacio muestral, sea $n(E)$ el número de veces que $E$ ocurre, y defina $f(E) = n(E)/n$. Muestre que f es una medida de probabilidad. (Es decir, satisface los Axiomas de la Probabilidad.)

\item Si $P(E) = 0.9$ y $P(F) = 0.8$, muestre que $P(E \cap F ) \ge 0.7.$ En general, pruebe la \textit{desigualdad de Bonferroni}, 

$$\Pr(E \cap F) \ge\Pr(E) +\Pr(F) - 1$$

\item  Sea $(\Omega, S,\Pr)$ un espacio de probabilidad donde $S$ es una $\sigma$-algebra de subconjuntos de $\Omega$ y $\Pr$ es una medida de probabilidad que asigna probabilidad $p(> 0)$ a cada uno de los puntos de $\Omega$.

\begin{itemize}
	\item Muestre que $\Omega$ debe tener un número finito de puntos. (\textbf{Sugerencia: muestre que $\Omega$ no puede tener más de $1/p$ puntos}.)
	\item Muestre que si $n$ es el número de puntos de $\Omega$, entonces p debe ser $1/n$.
\end{itemize}

		\item Sean $\mathbb{P}_1, \dots, \mathbb{P}_n$ medidas de probabilidad sobre $(\Omega, \mathcal{F})$ y $\alpha_1,\dots,\alpha_n$ números no negativos tales que $\sum_{i=1}^{n} \alpha_i = 1$. Demuestre que la combinación convexa
\[
\mathbb{P} = \sum_{i=1}^{n} \alpha_i \mathbb{P}_i
\]
es una medida de probabilidad.

\item Supongamos que tenemos dos dados honestos. Resuelva los siguientes incisos:
\begin{enumerate}
	\item Calcule la probabilidad de que la suma del resultado de ambos dados sea igual a ocho.
	\item Obtenga la probabilidad de que el resultado del primer dado sea menor al resultado del segundo.
	\item Calcule la probabilidad de que al menos sale un '6'.
\end{enumerate} 

\item Sean $A,B$ y $C$ tres eventos en el espacio muestral $\Omega$. Suponga que
\begin{itemize}
	\item $A \cup B \cup C = \Omega$.
	\item $\mathbb{P}(A) = \frac{1}{2}$.
	\item $\mathbb{P}(B) = \frac{2}{3}$.
	\item $\mathbb{P}(A \cup B) = \frac{5}{6}$.
\end{itemize}
Resuelva los siguientes incisos:
\begin{enumerate}
	\item Calcule $\mathbb{P}(A \cap B)$.
	\item ¿$A,B$ y $C$ forman una partición de $\Omega$?
	\item Encuentre $\mathbb{P}(C - (A \cup B))$.
	\item Si $\mathbb{P}(C \cap (A \cup B)) = \frac{5}{12}$, halle $\mathbb{P}(C)$.
\end{enumerate}

\item Se escoge un número $a$ al azar dentro del intervalo $(-1,1)$ ¿Cuál es la probabilidad de que la ecuación cuadrática $ax^2 + x + 1 = 0$ tenga dos raíces reales?

\item Un modelo de una ruleta puede construirse tomando un espacio de probabilidad uniforme sobre una circunferencia de radio $1$, de manera que la probabilidad de que el apuntador caiga en una arco de longitud $s$ es $s/2\pi$. Suponga que el círculo se divide en 37 zonas numeradas $1, 2, . . . , 37$. Calcule la probabilidad de que la ruleta caiga en una zona par.

\item Suponga que un experimento se realiza n veces. Para cualquier evento $E$ del espacio muestral, sea $n(E)$ el número de veces que $E$ ocurre, y defina $f(E) = n(E)/n$. Muestre que $f$ es una medida de probabilidad. (Es decir, satisface los Axiomas de la Probabilidad.)

\item \textbf{Desigualdad de Bonferroni:}
Sean $E$ y $F$ conjuntos arbitrarios en un espacio de probabilidad. Pruebe que
\[
\mathbb{P}(E \cap F) \ge \mathbb{P}(E) + \mathbb{P}(F) - 1
\]
Asimismo, verifique que si $\mathbb{P}(E) = 0.9$ y $\mathbb{P}(F) = 0.8$, entonces $P(E \cap F ) \ge 0.7.$  

\item Un estudio sobre el número de sismos mayores a 5 grados en escala Richter que ocurren al año en la Ciudad de México demuestra que la probabilidad de tener $x+1$ sismos en un año es $\frac{1}{3}$ de la probabilidad de tener $x$ sismos en un año. ¿Cuál es la probabilidad de que en un año ocurran dos o más sismos?

\subsection*{Jordán}

\item Pruebe \textit{la desigualdad de Boole.}

$$ P(\bigcup_n A_n) \le \sum_n \mathbb{P}(A_n)$$

\item Considere una sucesión de eventos $A_1, A_2,...$ Muestre que si $\mathbb{P}(A_i) = 1,$ para todo $i>1$, entonces $\mathbb{P}(\cap_{i = 1}^{\infty} A_i) = 1.$

\item Un experimento aleatorio consiste en lanzar un dado hasta que salga un 6. En ese momento el experimento termina ¿Cuál es el espacio muestral del experimento? Sea $E_n$ el evento de que $n$ lanzamientos son necesarios para completar el experimento ¿Qué puntos del espacio muestral pertenecen a $E_n$? ¿Qué es $(\cup_{n=1}^{\infty} E_n)^c$?

\item Se lanzan un par de dados. 
\begin{enumerate}
	\item ¿Cúal es la probabilidad de que la salida del segundo dado sea mayor que la del primero?
	\item Se lanzan un par de dados hasta que sale un 5 ó un 7 ¿Cuál es la probabilidad de que salga primero el 5?
\end{enumerate}

\item Una escuela ofrece clases de tres idiomas: inglés, francés y alemán. Las clases están abiertas a todos los 100 estudiantes en la escuela. Hay 28 alumnos en la clase de inglés, 26 en la de francés, y 16 en la de alemán. Hay 12 estudiantes llevando inglés y francés, 4 llevando ingléss y alemánn, y 6 cursando alemán y francéss. Hay además 3 estudiantes llevando los 3 idiomas.
\begin{enumerate}
	\item Si se elige al azar un estudiante, ¿cuál es la probabilidad de que no esté llevando ninguno de los idiomas?
	\item Si se selecciona aleatoriamente un estudiante, ¿cuál es la probabilidad de que esté estudiando un idioma exactamente?
	\item Si 2 estudiantes se seleccionan al azar, calcule la probabilidad de que al menos uno esté llevando una clase de idiomas.
\end{enumerate}

\item Un sistema consiste en 5 componentes, las cuales están trabajando apropiadamente o están fallando. Considere el experimento de observar el estado de cada una de las componentes y sea la salida del experimento un vector de la forma $(x_1,x_2,x_3,x_4,x_5)$, donde $x_i$ es 0 ó 1, dependiendo de que la componente \textit{i-ésima} $(i = 1, . . . , 5)$ ha fallado o esté trabajando, respectivamente.
\begin{enumerate}
	\item ¿Cuántos elementos hay en el espacio muestral?
	\item Suponga que el sistema trabaja si 1 y 2 trabajan, o si las componentes 3 y 4 están ambas trabajando, o bien, si las componentes 1, 3 y 5 están todas trabajando. Sea $W$ el evento \textit{el sistema funciona}. Especifique todos los elementos de $W$.
	\item Sea A el evento que denota cuando las componentes 4 y 5 \textit{han fallado} ¿Cuántos elementos tiene el evento A?
\end{enumerate}

\item Determine el número de distintos arreglos $(x_1, . . . , x_n)$, tales que $x_i$ es 0 ó 1 y

$$ \sum_{i=1}^{n} x_i \ge k $$
\end{enumerate}

\section*{Soluciones}
\begin{enumerate}
\item Definimos, los siguientes conjuntos
\[ B_1 = A_1 \]
\[ B_2 = A_2 - A_1\]
\[ ... \]
\[ B_n = \bigcup_{k=1}^{n-1} A_k, \hspace{0.2cm} n =2,3,...\]

Luego, utilizaremos las siguientes 3 propiedades para demostrar la proposición final

\begin{enumerate}
	\item \[ \bigcup_{n=1}^\infty A_n = \bigcup_{n=1}^\infty B_n\]
	\item \[ B_n \cap B_m = \empty, \, n \neq m\]
	\item \[ B_n \subseteq A_n\]
\end{enumerate}

Es así como concluimos lo siguiente

\[ \mathbb{P}(\bigcup_{n=1}^\infty A_n) \overset{a}{=} \mathbb{P}(\bigcup_{n=1}^\infty B_n) = \sum_{n=1}^\infty \mathbb{P}(B_n) \overset{c}{\le} \sum_{n=1}^\infty \mathbb{P}(A_n)\]

\item Utilizando la proposición anterior

\[ \mathbb{P}(\bigcap_{i=1}^\infty A_i) = 1- \mathbb{P}(\bigcup_{i=1}^\infty A_i^c) \ge 1- \sum_{i=1}^n \mathbb{P}(A_i^c) = 1\]

\item Notemos que 
\begin{enumerate}
	\item El espacio muestral del experimento está dado por:
	\[ \Omega = \{ (6), (1,6),...,(5,6),(1,1,6),...\}\]
	
	\item Pertenecen todos aquellos eventos en donde no cayó un 6 hasta el n-ésimo tiro.
	
	\item Si $\cup_{n=1}^\infty E_n$ significa que eventualmente cae el dado, entonces $(\cup_{n=1}^\infty E_n)^c$ significa que nunca caerá.
\end{enumerate}

\item Sean $d_1,d_2$ las posibles salidas del dado 1 y del dado 2. Nuestro evento de interés está dado por: \[ A = \{d_1,d_2 \in \Omega \,|\, d_2 > d_1 \} = \{(2,1),(3,2),(3,1),(4,1),...,(4,3),(5,1),...,(5,4),(6,1),...,(6,5)\} \]

Luego, por el enfoque de probabilidad clásica, obtenemos

\[ \mathbb{P}(A) = \frac{|A|}{|\Omega|} = \frac{15}{6 \cdot 6}\]

\item Sean $I,F,A$ los eventos donde un alumno participa en la clase de inglés, francés, y alemán respectivamente. Notemos que

\begin{enumerate}
	\item \[ \mathbb{P}(I^c \cap\ F^c \cap A^c)= 1 - \mathbb{P}(I \cup F \cup A)\]
	\[= 1- [ \mathbb{P}(I) + \mathbb{P}(F) + \mathbb{P}(A) - \mathbb{P}(I \cap F) - \mathbb{P}(I \cap A) - \mathbb{P}(F \cap A) + \mathbb{P}(I \cap F \cap A)]\] 
	\[= 28 + 26 +16 - 12 - 4-6 +3 = 51\]
	
	\item Sea $B$ el evento tal que un estudiante esté llevando al menos 1 idioma, entonces
	\[ \mathbb{P}(B) = \frac{15+11+9}{100} = \frac{35}{100}\]
	
	\item Sea $C$ el evento donde al menos 1 estudiante, de dos elegidos aleatoriamente, esté llevando una clase de idiomas
	\[ \mathbb{P}(C)= 1- \mathbb{P}(C^c) = 1 - \frac{49}{100} \cdot \frac{48}{99}\] 
\end{enumerate}

\item \begin{enumerate}
	\item El número de elementos totales en el espacio muestral $\Omega$ está dado por $|\Omega| = 2^5.$
	\item Sea $W$ el evento donde el sistema trabaja,
	\[ W = \{X_1, X_2, X_3, X_4, X_5 \in \Omega | \]
	\[ (X_1, X_2, X_3, X_4, X_5) = (1,1,X_3, X_4, X_5)\]
	\[ (X_1, X_2, X_3, X_4, X_5) = (X_1,X_2,1, 1, X_5)\]
	\[ (X_1, X_2, X_3, X_4, X_5) = (1,X_2,1, X_4, X_5) \}\]
	
	\item El número de elementos de $A$ está dado por $|A| = 2^3.$
\end{enumerate}

\item Supongamos primero que $k=2$, entonces podemos componer una tupla $(1,1,X_3,...,X_n),$ y el número de distintos arreglos que se pueden formar es de $2^{n-2}.$

Suponiendo ahora que $k=3$, entonces podemos hacer una tupla $(1,1,1,x_4,...,X_n),$ y el número de distintos arreglos que se pueden formar es de $2^{n-3}.$

Luego, es fácil concluir que el número de distintos arreglos para que se cumpla la condición está dado por $2^{n-k}.$







\end{enumerate}

\subsubsection*{Textos de apoyo.} 
\citeAY{Bartle:1966}; \citeAY{Blitzstein&Hwang:2014}; \citeAY{Hoel&etal:1971}; \citeAY{Rincon:2014}; \citeAY{Ross:2014}.
