\documentclass[10pt,a4paper]{article}
\usepackage[utf8]{inputenc}
\usepackage[T1]{fontenc}
\usepackage{amsmath}
\usepackage{amsfonts}
\usepackage{amssymb}
\usepackage{graphicx}
\usepackage[left=2.00cm, right=2.00cm]{geometry}
\usepackage[spanish]{babel}
\usepackage{geometry}
\usepackage{url}
\usepackage{enumerate}
\usepackage{mathtools}
\usepackage{graphicx}
\usepackage{bbm}
\usepackage{dsfont}
\usepackage{array}
\newcolumntype{P}[1]{>{\centering\arraybackslash}p{#1}}
\usepackage{multirow}
\usepackage{graphicx}
\usepackage{xcolor, colortbl}
\usepackage{lscape}
\usepackage{array, multirow, multicol}
\spanishdecimal{.}

\definecolor{skyblue6}{rgb}{.2, .6, .8}
\definecolor{gris claro}{gray}{0.95}

\title{Cálculo de Probabilidades I \\
	\large Laboratorio 2\\ Instituto Tecnológico Autónomo de México}

\author{David Isaac López Romero}
\date{\today}
\begin{document}
	
	\maketitle
	
	\section{Medida de probabilidad}
	
	\begin{enumerate}
		
		\item Sean $\mathbb{P}_1, \dots, \mathbb{P}_n$ medidas de probabilidad sobre $(\Omega, \mathcal{F})$ y $\alpha_1,\dots,\alpha_n$ números no negativos tales que $\sum_{i=1}^{n} \alpha_i = 1$. Demuestre que la combinación convexa
		\[
		\mathbb{P} = \sum_{i=1}^{n} \alpha_i \mathbb{P}_i
		\]
		es una medida de probabilidad.
		
		\item Supongamos que tenemos dos dados honestos. Resuelva los siguientes incisos:
		\begin{enumerate}
			\item Calcule la probabilidad de que la suma del resultado de ambos dados sea igual a ocho.
			\item Obtenga la probabilidad de que el resultado del primer dado sea menor al resultado del segundo.
			\item Calcule la probabilidad de que al menos sale un '6'.
		\end{enumerate} 
		
		\item Sean $A,B$ y $C$ tres eventos en el espacio muestral $\Omega$. Suponga que
		\begin{itemize}
			\item $A \cup B \cup C = \Omega$.
			\item $\mathbb{P}(A) = \frac{1}{2}$.
			\item $\mathbb{P}(B) = \frac{2}{3}$.
			\item $\mathbb{P}(A \cup B) = \frac{5}{6}$.
		\end{itemize}
		Resuelva los siguientes incisos:
		\begin{enumerate}
			\item Calcule $\mathbb{P}(A \cap B)$.
			\item ¿$A,B$ y $C$ forman una partición de $\Omega$?
			\item Encuentre $\mathbb{P}(C - (A \cup B))$.
			\item Si $\mathbb{P}(C \cap (A \cup B)) = \frac{5}{12}$, halle $\mathbb{P}(C)$.
		\end{enumerate}
	
		\item Se escoge un número $a$ al azar dentro del intervalo $(-1,1)$ ¿Cuál es la probabilidad de que la ecuación cuadrática $ax^2 + x + 1 = 0$ tenga dos raíces reales?
		
		\item Un modelo de una ruleta puede construirse tomando un espacio de probabilidad uniforme sobre una circunferencia de radio $1$, de manera que la probabilidad de que el apuntador caiga en una arco de longitud $s$ es $s/2\pi$. Suponga que el círculo se divide en 37 zonas numeradas $1, 2, . . . , 37$. Calcule la probabilidad de que la ruleta caiga en una zona par.
		
		\item Suponga que un experimento se realiza n veces. Para cualquier evento $E$ del espacio muestral, sea $n(E)$ el número de veces que $E$ ocurre, y defina $f(E) = n(E)/n$. Muestre que $f$ es una medida de probabilidad. (Es decir, satisface los Axiomas de la Probabilidad.)
		
		\item \textbf{Desigualdad de Bonferroni:}
		Sean $E$ y $F$ conjuntos arbitrarios en un espacio de probabilidad. Pruebe que
		\[
		\mathbb{P}(E \cap F) \ge \mathbb{P}(E) + \mathbb{P}(F) - 1
		\]
		Asimismo, verifique que si $\mathbb{P}(E) = 0.9$ y $\mathbb{P}(F) = 0.8$, entonces $P(E \cap F ) \ge 0.7.$  
		
		\item Un estudio sobre el número de sismos mayores a 5 grados en escala Richter que ocurren al año en la Ciudad de México demuestra que la probabilidad de tener $x+1$ sismos en un año es $\frac{1}{3}$ de la probabilidad de tener $x$ sismos en un año. ¿Cuál es la probabilidad de que en un año ocurran dos o más sismos?
		
		
		
	\end{enumerate}
	
\end{document}
