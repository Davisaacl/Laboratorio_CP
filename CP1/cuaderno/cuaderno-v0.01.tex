% ==============================================================================
\documentclass[11pt]{article}
% ==============================================================================

% ++++++++++++++++++++++++++++++++++++++++++++++++
\newcommand{\raiz}{../..}
\usepackage{\raiz/tex/preambulo, \raiz/tex/definiciones}
\newcommand{\home}{\raiz/CP1/}
% ++++++++++++++++++++++++++++++++++++++++++++++++

% ++++++++++++++++++++++++++++++++++++++++++++++++++++++++++++++++++++++++++++++
% ++++++++++++++++++++++++++++++++++++++++++++++++++++++++++++++++++++++++++++++

% ##########################################################
\def\doctitle{Laboratorio}
\def\topictitle{Cuaderno de Ejercicios}
\def\theauthor{López, Linares y Barrios}
\def\course{\textsc{Cálculo de Probabilidades I}}
\def\period{versión \version}
\def\version{0.01}
% ##########################################################

% ==============================================================================
\begin{document}
% ==============================================================================

\title{\bf {
		\huge Cálculo de Probabilidades I \\[1ex] 
		\LARGE Cuaderno de Ejercicios \\[1ex]
		\Large Laboratorio
		} }
\author{\large David López, Jordán Linares y Ernesto Barrios}
\date{\today \\[1ex] \small Versión \version}

\maketitle

\tableofcontents

\clearpage \newpage \

\section*{Prefacio}
\addcontentsline{toc}{section}{Prefacio}

Hace dos años se iniciaron las sesiones de ejercicios que llaman \emph{laboratorios} de los cursos de Cálculo de Probabilidades I y II. Los encargados de los laboratorios fueron David I. López Romero y L. Jordán L. Linares Pérez, alternándose los cursos. Cada uno de ellos colectó los ejercicios para sus sesiones. Ahora hemos empezado a recuperar las respuestas de los ejercicios y resolveremos una selección de ellos. Este cuaderno es el resultado de ellos.

\medskip

\noindent Cualquier error que identifique, comentario y/o sugerencia serán bienvenido. Diríjalo a Ernesto Barrios\ \texttt{<ebarrios at itam.mx>}.

\bigskip

\begin{flushright}Ciudad de México, 2 de mayo de 2022 \end{flushright}

\newpage 

\section{Introducción}

\begin{enumerate}
	\item Sean $A$ y $B$ subconjuntos de $\Omega$. Demuestre que
	\[
	A \subseteq B \Longleftrightarrow B^c \subseteq A^c
	\]
	
	\item Demuestre que si $A$ y $B$ son conjuntos, entonces
	\[
	\mathcal{P}(A \cap B) = \mathcal{P}(A) \cap \mathcal{P}(B)
	\]
	
	
	\item Demuestre las leyes (o fórmulas) de De Morgan. Si $\{ A_i | i \in I\}$ es una colección arbitraria de subconjuntos de $\Omega$, entonces
	\begin{enumerate}
		\item $\Big( \bigcup_{i} A_i \Big)^c = \bigcap_{i} A_i^c$
		\item $\Big( \bigcap_{i} A_i \Big)^c = \bigcup_{i} A_i^c$
	\end{enumerate}
	
	
	\item Demuestre que si $\mathcal{F}$ es una $\sigma$-álgebra de subconjuntos de $\Omega$ si, y solo si, se satisfacen las siguientes propiedades:
	\begin{enumerate}
		\item $\emptyset \in \mathcal{F}$ 
		\item $A \in \mathcal{F} \Longrightarrow A^{c} \in \mathcal{F}$
		\item $(A_n)_{n=1}^{+\infty} \in \mathcal{F} \Longrightarrow \bigcap\limits_{n=1}^{\infty} A_n \in \mathcal{F}$
	\end{enumerate}
	
	
	
	
	\item Pruebe que el conjunto potencia de $\Omega \not= \emptyset$ arbitrario es una $\sigma$-álgebra.
	
	\item Sea $\Omega = \{a,b,c,d\}$ y sean $A=\{a,b\}$ y $B = \{b,c\}$. Defina la familia $\mathcal{A} = \{A,B\}$. Determine si $\mathcal{A}$ es $\sigma$-álgebra. Encuentre la mínima $\sigma$-álgebra que contiene a $\mathcal{A}$, que se define por:
	
	\[
	\sigma\{\mathcal{A}\} = \bigcap_i \{ \mathcal{F}_i | \mathcal{F}_i \supset \mathcal{A} \}
	\]
	
	\item Sean $\mathcal{F}_i$ para $i=1,\dots,n$ una colección de $\sigma$- álgebras. Defina a 
	\[
	\mathcal{F} = \bigcap_{i=1}^{n} \mathcal{F}_i
	\]
	Pruebe que $\mathcal{F}$ es $\sigma$-álgebra.
	
	(\textbf{Observación:} La demostración en versión infinita es análoga.)
	
	\item Sean $\mathcal{F}_1$ y $\mathcal{F}_2$ dos $\sigma$- álgebras de subconjuntos de $\Omega$. Pruebe que $\mathcal{F}_1 \cup \mathcal{F}_2$ no necesariamente es una $\sigma$-algebra. Para ello considere el espacio $\Omega = \{1,2,3\}$ y las $\sigma$-álgebras $\mathcal{F}_1 = \{ \emptyset, \{1\}, \{2,3\}, \Omega\}$ y $\mathcal{F}_2 = \{ \emptyset, \{1,2\}, \{3\}, \Omega\}$.
	
	
	\item Sea $\mathcal{F}$ una $\sigma$-álgebra de subconjuntos de $\Omega$. Pruebe que la colección $\mathcal{F}^C = \{ A^c | A \in \mathcal{F}\}$ es una $\sigma$-álgebra. Compruebe que $\mathcal{F}^C$ y $\mathcal{F}$ coinciden.
	
	
	
	\item (\textbf{Opcional}:) Sean $\Omega$ y $\Omega'$ conjuntos arbitrarios y $f:\Omega \rightarrow \Omega'$ una función. Si $B \subset \Omega'$, la imagen inversa de $B$ con respecto a $f$ será 
\[
f^{-1}(B) = \{ \omega \in \Omega | f(\omega ) \in B\}
\]
Si $\mathcal{C}$ es una familia de subconjuntos $\Omega'$, entonces
\[
f^{-1} (\mathcal{C}) = \{ f^{-1} (B) | B \in \mathcal{C}\}
\]
Demuestre:

\begin{enumerate}
	\item $f^{-1}(\Omega') = \Omega$.
	\item Si $B$ y $C$ son subconjuntos de $\Omega'$ entonces $f^{-1}(C-B)=f^{-1}(C)-f^{-1}(B)$. En particular, $f^{-1}(B^{c}) = [f^{-1}(B)]^{c}$ y $f^{-1}(\emptyset)= \emptyset$.
	\item Si $\{ B_i, i \in I\}$ es una familia arbitraria de subconjuntos de $\Omega'$, entonces
	\[
	f^{-1} \Big( \bigcup_{i} B_i \Big) = \bigcup_{i} f^{-1}(B_i) \quad \text{y} \quad f^{-1} \Big( \bigcap_{i} B_i \Big) = \bigcap f^{-1} (B_i)
	\]
	\item Si $\mathcal{F}'$ es una $\sigma$-álgebra de $\Omega'$, entonces la familia
	\[
	f^{-1} (\mathcal{F} ') = \{ f^{-1} (B) | B \in \hspace{1mm} \mathclap{F}\hspace{1.5mm} '\}
	\]
	es una $\sigma$-álgebra de $\Omega$.
\end{enumerate}	

	
\end{enumerate}

\subsubsection*{Textos de apoyo.} 
\citeAY{Bartle:1966}; \citeAY{Blitzstein&Hwang:2014}; \citeAY{Hoel&etal:1971}; \citeAY{Rincon:2014}; \citeAY{Ross:2014}.


% -----------------------------------------
\addcontentsline{toc}{section}{Referencias}
\bibliography{\raiz/ref/probabilidad}
\bibliographystyle{chicago}
% -----------------------------------------

\clearpage \newpage \

%%% Respuestas
\renewcommand{\thesubsection}{\arabic{subsection}}
\addcontentsline{toc}{section}{Respuestas}
\section*{Respuestas}

\renewcommand{\theenumi}{\thesubsection:\arabic{enumi}}
%\settocdepth{section} \setcounter{subsection}{0}
\subsection{Introducción}

\begin{enumerate}
\item  ---
\item  ---
\item  ---
\item  ---
\item  ---
\item  ---
\item  ---
\item  ---

\end{enumerate}


% ==============================================================================
\end{document}
% ==============================================================================
